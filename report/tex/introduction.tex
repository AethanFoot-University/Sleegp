\section{Introduction}

\subsection{Overview of Domain}

% Personal informatics

Personal informatics is a term used to refer to devices and software that help people gather
information about themselves, so they can reflect upon it and gain motivation to make changes to
their lifestyle and habits to improve their overall wellbeing. Personal informatics is used for
effectively motivating people to gain self-knowledge, change behaviours. % Gen's source here

The area of personal informatics has started to explode in popularity in recent years mainly due to
the increased availability and usability of affordable hardware. Consumer products such as the
FitBit and Apple Watch allow users to collect data on a wide variety of metrics including heart
rate, blood pressure, motion and many others. Products such as the Neuroon, a wearable EEG eye mask,
and Zeo Sleep Manager Pro, an EEG headband, allow the user to collect information on brain waves for
the purpose of sleep tracking.
% Validity of Consumer Activity Wristbands and Wearable EEG for Measuring Overall Sleep Parameters
% and Sleep Structure in Free-Living Conditions,

% Consumer Sleep Tracking Devices: A Critical Review in Digital Healthcare Empowering Europeans -
% write more here about other devices

Another factor that has contributed to the growth of personal informatics is the ubiquity of
smartphones, meaning that users have an ever-present device that allows them to collect and collate
data from their personal informatics hardware. Many personal informatics apps also add an element of
socially driven competition and gamification, driving users' motivation to continue to use them and
push their friends to also begin using this technology. In addition, there is a larger social force
pushing people to take steps to improve themselves. % Expand on this sentence

% Gamification!!

\subsection{Challenges}
Although personal informatics systems for wellbeing has been on the rise, it inherently presents flaws that need to
be taken into account. In a survey conducted by / where participants were regular personal informatics users it was revealed
that the most significant shortcomings of personal informatics systems supplied by commercial companies was a lack of understanding
for the end user's requirements and an absence of assistance and alerts for users who didn't meet their goals.
Apart from users who are familiar with personal informatics systems it is also important to consider the challenges
faced by the common user; a user who is new to using a personal informatics device. It was discovered by / that the main
challenge for personal informatics systems was the lack of motivation faced by the end user to continue to use the system.
These challenges need to be taken into account because these hinder the end user from improving themselves which is contradictory
to the goal of personal informatics systems.

\subsubsection{Privacy and Security of Data}

% Data has to be stored somewhere, it's often being collected through third-party apps, no way to
% ensure it isn't transmitted or stored elsewhere

% Scraping of data with Cambridge Analytica, Strava military base heatmap

\subsubsection{Health Risks}

One crucial problem in the realm of health is sleep deprivation. Sleep deprivation is defined by
\textcite{BMA:2018aa} as ``a lack of sufficient sleep resulting from disruption to the natural sleep
cycle''. This is important to highlight because as as opposed to fatigue, sleep deprivation isn't
subjective. In accordance to \textcite{Alhola:2007aa}, it was estimated that the main effect of
sleep deprivation was the reduction in cognitive performance. This includes: impaired attention;
longer delays in making decisions; poor quality of decisions and a reduction in long memory. This is
especially important to monitor for individuals who have high risk jobs. In 2010, passenger's lives
were lost when a plane overshot the runway by 600 meters; although concrete details haven't been
exposed, it was claimed that the accident unfolded due to the pilot's severe sleep deprivation
\parencite{BBC:2010aa}. Even in circumstances where the individual isn't responsible for people's
lives, a reduction in cognitive performance is still observed. Hence, the validity of this problem
is justified.
